\documentclass[12pt,a4paper,faculty=we,language=nl,doctype=article]{ugent-doc}
\usepackage[T1]{fontenc}
\usepackage[utf8]{inputenc}
\usepackage[]{graphicx}
\usepackage[dutch]{babel}
\usepackage{float}
\usepackage[most]{tcolorbox}
\usepackage{lipsum} % for placeholder text
\usepackage[colorlinks=true,citecolor=ugentblue, urlcolor=ugentblue]{hyperref}
\usepackage[backend=biber,style=numeric, sorting=anyt]{biblatex}
\addbibresource{references.bib}
\graphicspath{{./figures/}}
\usepackage{csquotes}
\usepackage[parfill]{parskip} 
\usepackage{ulem}
\renewcommand{\ULthickness}{2pt}
\usepackage{libertine}
\usepackage{lmodern}
\usepackage{titlesec}

\titlespacing{\section}{0pt}{1.8ex plus 0.2ex minus 0.2ex}{1ex}
\titlespacing{\subsection}{0pt}{0.8ex plus 0.1ex minus 0.1ex}{0.8ex}


\thetitle{\uline{\color{ugentblue}Optimalisatie van het ClassiCOL algoritme}}

% The first (top) infobox at bottom of titlepage
\infoboxa{\bfseries\large Vak: Computationele Biologie}

% The second infobox at bottom of titlepage
\infoboxb{Lesgevers: 
	\begin{tabular}[t]{l}
		Prof.\ Dr.\ Peter Dawyndt\\ % note syntax 'short space'
		Simon Van de Vyver\\
		Benjamin Rombaut
	\end{tabular}
}

% The third infobox at bottom of titlepage
\infoboxc{Student: 
	\begin{tabular}[t]{lll}
		Thomas Dejaeghere & 02107294 & thomas.dejaeghere@ugent.be\\
	\end{tabular}
}

% The last (bottom) infobox at bottom of titlepage
\infoboxd{Academisch jaar: 2024--2025} % note dash, not hyphen

\begin{document}
	\maketitle
	\renewcommand{\ULthickness}{1pt}
	\begin{tcolorbox}[
		colback=white,
		colframe=cyan!70!black,
		boxrule=0.8pt,
		arc=3mm,
		title=\textbf{Abstract},
		fonttitle=\bfseries,
		coltitle=black,
		enhanced
		]
		\textbf{Achtergrond:}
		
		\textbf{Resultaten:}
		
		\textbf{Conclusies:} 
		
	\end{tcolorbox}
	
	\section{Introductie}
	
	\section{Methoden}
	
	
	\section{Resultaten}
	\begin{tcolorbox}[
		colback=white,
		colframe=cyan!70!black,
		boxrule=0.8pt,
		arc=3mm,
		title=\textbf{PC-specificaties},
		fonttitle=\bfseries,
		coltitle=black,
		enhanced
		]
		De benchmarks werden uitgevoerd op de volgende machine:
		
		\begin{itemize}
			\item \textbf{Model}: Lenovo Yoga Pro 7
			\item \textbf{Processor:} 13\textsuperscript{th} Gen Intel\textsuperscript{\textregistered} Core\textsuperscript{TM} i7-13700H @ 2.40 GHz (14 cores)
			\item \textbf{Geheugen (RAM):} 32,0 GB geïnstalleerd (29,7 GB bruikbaar)
			\item \textbf{Besturingssysteem:} Windows 11 Pro (64-bit)
		\end{itemize}
		
	\end{tcolorbox}
	
	
	\section{Conclusie}
	
	\section*{Appendix A}

	\nocite{openai2024chatgpt}
	\printbibliography
\end{document}